\subsubsection{Software Attributes}
\paragraph{Security}
\mbox{}\\
The user data needs to be secure in order to protect the users. By implementing a secure database in which user data is stored as well as a firewall to protect the server itself we can ensure that the system will remain secure. Another aspect of the security required by this system is password encryption. Due to some of the information stored in the user files being somewhat sensitive data (routes walked and common locations visited) it is crucial for us to keep these records secure and safe.

\paragraph{Portability}
\mbox{}\\
The system is inherently designed to be very portable. The user data can be entered regardless of the hardware device being used. This means that users can register on iPhone or Android and Admin users can create users if needed directly from the server. The software independence is also clear as again any device can be used to access and manage the user data.

\paragraph{Cohesion}
\mbox{}\\
The system is already a very coherent system. The user data will have to be used across a lot of the different subsystems such as Navigation and Rewards and as such it needs to be designed to be portable.

\subsubsection{Technology Choices}
For the users subsystem we will need a server to host the relevant tables. This server should have a database management software to allow for easy management. The server should be 
guarded by some form of firewall software to prevent any unwanted access. 

\subsubsection{Use Case Diagram}
	\begin{figure}[h!]
	\makebox[\textwidth][c]{\includegraphics[width=1.4\textwidth]{diagrams/users_subsystem/users_subsystem_draft.pdf}}
	\caption{Users Use Case}
	\end{figure}
	
\subsubsection{Class Diagram}
	\begin{figure}[h!]
	\makebox[\textwidth][c]{\includegraphics[width=1.4\textwidth]{diagrams/users_subsystem/users_class_diagram.pdf}}
	\caption{Users Class Diagram}
	\end{figure}