Perfomance Requirements
NavUP System
Performance requirements are required by every system, to ensure that the performance concerns and possible hiccups that are likely to come up subsequently would be well noted ahead of time, so it can be significantly dealt with. The NavUP system is non-exclusive, and thus has the following performance requirements:

Response time 
-	The response time of a system which relates to how quickly requests need to be processed, and thus the maximum satisfactory response time for most the user-computer/device interaction such as NavUP, needs to be maximised as much as possible. 
-	
-	Therefore, a key factor is to ensure that the system’s traffic is either minimised or managed in such a way so as to maximise the speed at which it can respond to user requests such as searching for a location on campus. 
-	Factors such as low coupling can increase response time, as less dependency amongst subsystems will automatically increase the speed of operation within a subsystem.

Throughput 
-	This relates to the amount of data the system can handle per time, i.e. the workload at which the response time is to be met. Here, procedures must be put in place to ensure fewer arrivals of data to the system to minimise the traffic and increase performance of the system. For instance, with about 30 000 individuals accessing the campus grounds and utilities, one would have to approximate the number of requests that would be made to the system, such as logins, registrations, and the main reason of the app –making requests of locations to navigate to. This is to ensure there are no lags, so as to frustrate users from using the application
-	Should be able to register and log users in without lagging, irrespective of how many users are using it
